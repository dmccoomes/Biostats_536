\documentclass[]{article}
\usepackage{lmodern}
\usepackage{amssymb,amsmath}
\usepackage{ifxetex,ifluatex}
\usepackage{fixltx2e} % provides \textsubscript
\ifnum 0\ifxetex 1\fi\ifluatex 1\fi=0 % if pdftex
  \usepackage[T1]{fontenc}
  \usepackage[utf8]{inputenc}
\else % if luatex or xelatex
  \ifxetex
    \usepackage{mathspec}
  \else
    \usepackage{fontspec}
  \fi
  \defaultfontfeatures{Ligatures=TeX,Scale=MatchLowercase}
\fi
% use upquote if available, for straight quotes in verbatim environments
\IfFileExists{upquote.sty}{\usepackage{upquote}}{}
% use microtype if available
\IfFileExists{microtype.sty}{%
\usepackage{microtype}
\UseMicrotypeSet[protrusion]{basicmath} % disable protrusion for tt fonts
}{}
\usepackage[margin=1in]{geometry}
\usepackage{hyperref}
\hypersetup{unicode=true,
            pdftitle={Homework 7},
            pdfauthor={David Coomes},
            pdfborder={0 0 0},
            breaklinks=true}
\urlstyle{same}  % don't use monospace font for urls
\usepackage{graphicx,grffile}
\makeatletter
\def\maxwidth{\ifdim\Gin@nat@width>\linewidth\linewidth\else\Gin@nat@width\fi}
\def\maxheight{\ifdim\Gin@nat@height>\textheight\textheight\else\Gin@nat@height\fi}
\makeatother
% Scale images if necessary, so that they will not overflow the page
% margins by default, and it is still possible to overwrite the defaults
% using explicit options in \includegraphics[width, height, ...]{}
\setkeys{Gin}{width=\maxwidth,height=\maxheight,keepaspectratio}
\IfFileExists{parskip.sty}{%
\usepackage{parskip}
}{% else
\setlength{\parindent}{0pt}
\setlength{\parskip}{6pt plus 2pt minus 1pt}
}
\setlength{\emergencystretch}{3em}  % prevent overfull lines
\providecommand{\tightlist}{%
  \setlength{\itemsep}{0pt}\setlength{\parskip}{0pt}}
\setcounter{secnumdepth}{0}
% Redefines (sub)paragraphs to behave more like sections
\ifx\paragraph\undefined\else
\let\oldparagraph\paragraph
\renewcommand{\paragraph}[1]{\oldparagraph{#1}\mbox{}}
\fi
\ifx\subparagraph\undefined\else
\let\oldsubparagraph\subparagraph
\renewcommand{\subparagraph}[1]{\oldsubparagraph{#1}\mbox{}}
\fi

%%% Use protect on footnotes to avoid problems with footnotes in titles
\let\rmarkdownfootnote\footnote%
\def\footnote{\protect\rmarkdownfootnote}

%%% Change title format to be more compact
\usepackage{titling}

% Create subtitle command for use in maketitle
\providecommand{\subtitle}[1]{
  \posttitle{
    \begin{center}\large#1\end{center}
    }
}

\setlength{\droptitle}{-2em}

  \title{Homework 7}
    \pretitle{\vspace{\droptitle}\centering\huge}
  \posttitle{\par}
    \author{David Coomes}
    \preauthor{\centering\large\emph}
  \postauthor{\par}
      \predate{\centering\large\emph}
  \postdate{\par}
    \date{11/17/2019}

\usepackage{booktabs}
\usepackage{longtable}
\usepackage{array}
\usepackage{multirow}
\usepackage{wrapfig}
\usepackage{float}
\usepackage{colortbl}
\usepackage{pdflscape}
\usepackage{tabu}
\usepackage{threeparttable}
\usepackage{threeparttablex}
\usepackage[normalem]{ulem}
\usepackage{makecell}
\usepackage{xcolor}

\begin{document}
\maketitle

\paragraph{Question 1}\label{question-1}

\paragraph{(a)}\label{a}

\textbf{Table 1:} Odds ratios and 95\% confidence intervals for the
effect of x-ray exposure adjusting for year of birth (using -10 for
1944).

\begin{table}[H]
\centering
\begin{tabular}{l|r|r|r|r}
\hline
  & Estimate & 95\% CI Upper & 95\% CI Lower & P-value\\
\hline
YOB:1944 & 2.46 & 1.80 & 3.36 & 0.00\\
\hline
YOB:1948 & 2.11 & 1.71 & 2.60 & 0.00\\
\hline
YOB:1952 & 1.81 & 1.59 & 2.06 & 0.00\\
\hline
YOB:1956 & 1.55 & 1.38 & 1.75 & 0.00\\
\hline
YOB:1960 & 1.33 & 1.10 & 1.62 & 0.00\\
\hline
YOB:1964 & 1.14 & 0.85 & 1.53 & 0.38\\
\hline
\end{tabular}
\end{table}

\paragraph{(b)}\label{b}

\paragraph{(i)}\label{i}

\textbf{Table 2:} Odds ratios and 95\% confidence intervals for the
effect of x-ray exposure adjusting for year of birth (using -20 for
1944).

\begin{table}[H]
\centering
\begin{tabular}{l|r|r|r|r}
\hline
  & Estimate & 95\% CI Upper & 95\% CI Lower & P-value\\
\hline
YOB:1944 & 2.46 & 1.80 & 3.36 & 0.00\\
\hline
YOB:1948 & 2.11 & 1.71 & 2.60 & 0.00\\
\hline
YOB:1952 & 1.81 & 1.59 & 2.06 & 0.00\\
\hline
YOB:1956 & 1.55 & 1.38 & 1.75 & 0.00\\
\hline
YOB:1960 & 1.33 & 1.10 & 1.62 & 0.00\\
\hline
YOB:1964 & 1.10 & 0.80 & 1.51 & 0.56\\
\hline
\end{tabular}
\end{table}

\paragraph{(ii)}\label{ii}

The odds ratios are the same for both models except for the OR for those
born in 1964. This model is not a reparamaterization - there is just one
variable that is coded differently. Both models include all of the same
variables.

\paragraph{(c)}\label{c}

\textbf{Table 3:} Odds ratios and 95\% confidence intervals for the
effect of x-ray exposure adjusting for year of birth using a quadratic
effect modification model (using -10 for 1944).

\begin{table}[H]
\centering
\begin{tabular}{l|r|r|r|r}
\hline
  & Estimate & 95\% CI Upper & 95\% CI Lower & P-value\\
\hline
YOB:1944 & 4.59 & 2.43 & 8.65 & 0.00\\
\hline
YOB:1948 & 2.51 & 1.93 & 3.27 & 0.00\\
\hline
YOB:1952 & 1.70 & 1.48 & 1.96 & 0.00\\
\hline
YOB:1956 & 1.43 & 1.24 & 1.64 & 0.00\\
\hline
YOB:1960 & 1.48 & 1.19 & 1.84 & 0.00\\
\hline
YOB:1964 & 1.90 & 1.11 & 3.22 & 0.02\\
\hline
\end{tabular}
\end{table}

\paragraph{(d)}\label{d}

\textbf{Table 4:} Odds ratios and 95\% confidence intervals for the
effect of x-ray exposure adjusting for year of birth using a quadratic
effect modification model (using -20 for 1944).


\end{document}
